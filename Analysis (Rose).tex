\documentclass{book}
\usepackage[utf8]{inputenc}
\usepackage[english]{babel}
 \usepackage[utf8]{inputenc}
\usepackage{amsmath}
\usepackage{amsthm}
\usepackage{amssymb}
\usepackage{natbib}
\usepackage{graphicx}
\usepackage{chngcntr}
\makeatletter
\def\th@plain{%
  \thm@notefont{}% same as heading font
  \itshape % body font
}
\def\th@definition{%
  \thm@notefont{}% same as heading font
  \normalfont % body font
}
\makeatother

\usepackage{notes2bib}
\usepackage{imakeidx}
\makeindex
\title{Elementary Analysis Notes}
\author{Yuxun LU}
\date{February 2017}

\theoremstyle{definition}
\newtheorem{definition}{Definition}[section]
\newtheorem{theorem}[definition]{Theorem}
\newtheorem{corollary}[definition]{Corollary}
\newtheorem{lemma}[definition]{Lemma}
\begin{document}
\maketitle
\section*{Preface}
This is the note of ``Elementary Analysis" (written by Kenneth A. Ross). The main content is the definitions, theorems corollaries and lemmas (some of them may appear in the exercises). This note is written by Yuxun LU from NAIST and under the Creative Commons License.
\tableofcontents
\mainmatter
\chapter{Introduction}
\section{The Set of Natural Numbers}

\begin{definition}[The Set of Natural Numbers]
The set of all positive integers is denoted by $\mathbb{N}$. Each positive integer $n$ has a successor.
\\ Properties (\textit{Peano Axioms} or \textit{Peano Postulates}):
\\ \textbf{N1.} 1 belongs to $\mathbb{N}$.
\\ \textbf{N2.} If $n$ belongs to $\mathbb{N}$, then its successor $n+1$ belongs to $\mathbb{N}$.
\\ \textbf{N3.} 1 is not the successor of any element in $\mathbb{N}$.
\\ \textbf{N4.} If $n$ and $m$ in $\mathbb{N}$ have the same successor, then $n=m$.
\\ \textbf{N5.} A subset of $\mathbb{N}$ which contains 1, and which contains $n+1$ whenever it contains $n$, must equal $\mathbb{N}$.
\end{definition}
\begin{definition}[Mathematical Induction]
Let $P_1, P_2, P_3, ...$ be a list of statements or propositions that may or may not be true. The principle of mathematical induction asserts all the statements $P_1, P_2, P_3, ...$ are true provided
\\ \textbf{I1} $P_1$ is true,
\\ \textbf{I2} $P_{n+1}$ is true whenever $P_n$ is true.
\end{definition}
\section{The Set of Rational Numbers}
\begin{definition}[algebraic number]
A number is called an \textit{algebraic number} if it satisfies a polynomial equation
\begin{equation*}
    c_nx^n + c_{n-1}x^{n-1} + ... + c_1x + c_0 = 0
\end{equation*}
where the coefficients $c_0, c_1, ..., c_n$ are integers, $c_n \neq 0$ and $n \geq 1$.
\end{definition}
\begin{theorem}[Rational Zeros Theorem]
Suppose $c_0, c_1,...,c_n$ are integers and $r$ is a rational number satisfying the polynomial equation
\begin{equation}
    c_nx^n + c_{n-1}x^{n-1} + ... + c_1x + c_0 = 0
\end{equation}
where $n \geq 1, c_n \neq 0$ and $c_0 \neq 0$. Let $r = \frac{c}{d}$ where $c,d$ are integers having no common factors and $d \neq 0$. Then $c$ divides $c_0$ and $d$ divides $c_n$.
\end{theorem}
\begin{corollary}
Consider the polynomial equation
\begin{equation*}
    x^n + c_{n-1}x^{n-1} + ... + c_1x + c_0 = 0
\end{equation*}
where the coefficients $c_0, c_1, ..., c_{n-1}$ are integers and $c_0 \neq 0$. Any rational solution of this equation must be an integer that divides $c_0$.
\end{corollary}
\section{The Set of Real Numbers}
The basic algebraic operations in $\mathbb{Q}$ are addition and multiplication. Given a pair $a,b$ of rational numbers, the sum $a+b$ and the product $ab$ also represent rational numbers. Moreover, the following properties hold.
\\ \textbf{A1.} $a+(b+c)=(a+b)+c$ for all $a,b,c$. (\textit{associative laws})
\\ \textbf{A2.} $a+b=b+a$ for all $a,b$. (\textit{commutative laws})
\\ \textbf{A3.} $a+0=a$ for all $a$.
\\ \textbf{A4.} For each $a$, there is anelement $-a$ s.t. $a+(-a)=0$.
\\ \textbf{M1.} $a(bc)=(ab)c$ for all $a,b,c$. (\textit{associative laws})
\\ \textbf{M2.} $ab = ba$ for all $a,b$. (\textit{commutative laws})
\\ \textbf{M3.} $a \cdot 1 = a $ for all $a$.
\\ \textbf{M4.} Fro each $a \neq 0$, there is an element $a^{-1}$ s.t. $aa^{-1}=1$.
\\ \textbf{DL.} $a(b+c)=ab+ac$ for all $a,b,c$. (\textit{distributive law})
\\ A system that has more than one element and satisfies these nine properties are called a \textit{field}.
\\ The set $\mathbb{Q}$ also has an order strtucture $\leq$ satisfying
\\ \textbf{O1.} Given $a$ and $b$, either $a \leq b$ or $b \leq a$.
\\ \textbf{O2.} If $a \leq b$ and $b \leq a$, then $a=b$.
\\ \textbf{O3.} If $a \leq b$ and $b \leq c$, then $a \leq c$. (\textit{transitive law})
\\ \textbf{O4.} If $a \leq b$, then $a+c \leq b+c$.
\\ \textbf{O5.} If $a \leq b$ and $0 \leq c$ then $ac \leq bc$.
Property O3 is called the transitive law. This is the characteristic property of an ordering. A field with an ordering satisfying properties O1 through O5 is called an \textit{ordered field}.
\\ Real numbers, i.e., elements of $\mathbb{R}$, can be added together and multiplied together. That is, given real numbers $a$ and $b$, the sum $a+b$ and the product $ab$ also represent real numbers. Moreover, these operations satisfy the field properties A1 through A4, M1 through M4, and DL. The set $\mathbb{R}$ also has an order structure $\leq$ that satisfies properties O1 through O5. Thus, like $\mathbb{Q}, \mathbb{R}$ is an ordered field.
\begin{theorem}
The following are consequences of the field properties:
\\ \textbf{(i)} $a+c=b+c$ implies $a=b$;
\\ \textbf{(ii)} $a \cdot 0 = 0$ for all $a$;
\\ \textbf{(iii)} $(-a)b=-ab$ for all $a,b$;
\\ \textbf{(iv)} $(-a)(-b)=ab$ for all $a,b$;
\\ \textbf{(v)} $ac=bc$ and $c \neq 0$ imply $a=b$;
\\ \textbf{(vi)} $ab=0$ implies either $a=0$ or $b=0$ for $a,b,c \in \mathbb{R}$.
\end{theorem}
\begin{theorem}
The following are consequences of the properties of an ordered field:
\\ \textbf{(i)} If $a \leq b$, then $\-b \leq -a$;
\\ \textbf{(ii)} If $a \leq b$ and $c \leq 0$, then $bc \leq ac$;
\\ \textbf{(iii)} If $0 \leq a$ and $0 \leq b$, then $0 \leq ab$;
\\ \textbf{(iv)} $0 \leq a^2$ for all $a$;
\\ \textbf{(v)} $0 < 1$;
\\ \textbf{(vi)} If $0 < a$, then $0 < a^{-1}$;
\\\ \textbf{(vii)} If $0 < a < b$, then $0 < b^{-1} < a^{-1}$; for all $a,b,c \in \mathbb{R}$.
\\ Note $a<b$ means $a \leq b$ and $a \neq b$.
\end{theorem}
\begin{definition}[Absolute Value]
We define $|a| = a$ if $a \geq 0$ and $|a| = -a$ if $a \leq 0$.
\\ $|a|$ is called the \textit{absolute value} of $a$.
\end{definition}
\begin{definition}[the Distance]
For numbers $a$ and $b$ we define dist($a,b$) = $|a-b|$; dist($a,b$) represents the \textit{distance} between $a$ and $b$.
\end{definition}
\begin{theorem}
Basic Properties of the Absolute Value
\\ \textbf{(i)} $|a| \geq 0$ for all $a \in \mathbb{R}$.
\\ \textbf{(ii)} $|ab| = |a| \cdot |b|$ for all $a,b \in \mathbb{R}$.
\\ \textbf{(iii)} $|a+b| \leq |a| + |b|$ for all $a,b \in \mathbb{R}$.
\end{theorem}
\begin{corollary}
dist($a,c$) $\leq$ dist($a,b$) + dist($b,c$) for all $a,b,c \in \mathbb{R}$.
\end{corollary}
\begin{theorem}[Triangle Inequality]
$|a+b| \leq |a| + |b|$ for all $a,b$.
\end{theorem}
\section{The Completeness Axiom}
\begin{definition}
Let $S$ be a nonempty subset of $\mathbb{R}$.
\\ \textbf{(a)} If $S$ contains a largest element $s_0$ [that is, $s_0$ belongs to $S$ and $s \leq s_0$ for all $s \in S$], then we call $s_0$ the \textit{maximum} of $S$ and written $s_0 = $ max $S$.
\\ \textbf{(b)} If $S$ contains a smallest element, then we call the smallest element the $minimum$ of $S$ and write it as min $S$.
\end{definition}
\begin{definition}
Let $S$ be a nonempty subset of $\mathbb{R}$.
\\ \textbf{(a)} If a real number $M$ satsifies $s \leq M$ for all $s \in S$, then $M$ is called an \textit{upper bound} of $S$ and the set $S$ is said to be \textit{bounded above}.
\\ \textbf{(b)} If a real number $m$ satisfies $m \leq s$ for all $s \in S$, then $m$ is called a \textit{lower bound} of $S$ and the set $S$ is said to be \textit{bounded below}.
\\ \textbf{(c)} The set $S$ is said to be \textit{bounded} if it is bounded above and bounded below. Thus $S$ is bounded if there exist real numbers $m$ and $M$ s.t. $S \subseteq [m, M]$.
\end{definition}
\begin{definition}
Let $S$ be a nonempty subset of $\mathbb{R}$.
\\ \textbf{(a)} If $S$ is bounded above and $S$ has a least upper bound, then we will call it the \textit{supremum} of $S$ and denote it by sup $S$.
\\ \textbf{(b)} If $S$ is bounded below and $S$ has a greatest lower bound, then we will call it the \textit{infimum} of $S$ and denote it by inf $S$.
\end{definition}
Note that, unlike max $S$ and min $S$, sup $S$ and inf $S$ need not belong to $S$ and a set can have at most one maximum, minimum, supremum and infimum.
\\ If $S$ is bounded above, then $M$ = sup $S$ iff (i) $s \leq M$ for all $s \in S$ and (ii) whenever $M_1 < M$, there exists $s_1 \in S$ s.t. $s_1 > M_1$.
\begin{theorem}[Completeness Axiom]
Every nonempty subset $S$ of $\mathbb{R}$ that is bounded above has a least upper bound. In other words, sup $S$ exists and is a real number.
\end{theorem}
The completeness axiom does not hold for $\mathbb{Q}$.
\begin{corollary}
Every nonempty subset $S$ of $\mathbb{R}$ that is bounded below has a greatest lower bound inf $S$.
\end{corollary}
\begin{theorem}[Archimedean Property]
If $a > 0$ and $b > 0$, then for some positive integer $n$, we have $na > b$.
\end{theorem}
\begin{theorem}
Denseness of $\mathbb{Q}$.
\\ If $a,b \in \mathbb{R}$ and $a < b$, then there is a rational $r \in \mathbb{Q}$ s.t. $a < r < b$.
\end{theorem}
\section{The Symbols $+\infty$ and $-\infty$}
Let $S$ be any nonempty subset of $\mathbb{R}$. The symbols sup $S$ and inf $S$ always make sense. If $S$ is bounded above, then sup $S$ is a real number; otherwise sup $S$ = $+\infty$. If $S$ is bounded below, then inf $S$ is a real number; otherwise inf $S$ = $-\infty$. Moreover, we have inf $S$ $\leq$ sup $S$.

$[a, \infty)$ and $(-\infty, b]$ are called \textit{closed intervals} while $[a, b]$ is called \textit{\textbf{a} closed interval}.

Let $S$ be any nonempty subset of $\mathbb{R}$. The symbols $sup S$ and $inf S$ always make sense. If $S$ is bounded above, then $sup S$ is a real number; otherwise $supS=+\infty$. If $S$ is bounded below, then $inf S$ is a real number; otherwise $inf S= -\infty$. Moreover, we have $inf S \leq sup S$.

\bibliographystyle{unsrt}
% \bibliography{sample}

\end{document}
