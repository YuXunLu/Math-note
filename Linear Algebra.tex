\documentclass{article}
\usepackage[utf8]{inputenc}
\usepackage[english]{babel}
 
\usepackage{amsthm}
\makeatletter
\def\th@plain{%
  \thm@notefont{}% same as heading font
  \itshape % body font
}
\def\th@definition{%
  \thm@notefont{}% same as heading font
  \normalfont % body font
}
\makeatother
\usepackage{notes2bib}


\title{notes2bib - Integrating notes into the bibliography}
\author{ShareLaTeX Templates}
\date{March 2014}

\theoremstyle{definition}
\newtheorem{definition}{Definition}[section]

\begin{document}

\maketitle

This is a very basic example of using \texttt{notes2bib} package in order to put notes together with bibliography. 
Let us begin from citation of \LaTeX\ related references \cite{latexcompanion,knuthwebsite}. Next, is a very simple example of a bibliography note \bibnote{Note for the first example}. Afterward we can referee to famous Einstein's paper \cite{einstein}. Finally, there is next note \bibnote{Note for the second example} inside text. 

\begin{definition}[Field]
A set is called field if and only if.
\end{definition}

\bibliographystyle{unsrt}
\bibliography{sample}

\end{document}
